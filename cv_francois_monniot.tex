\documentclass[10pt,a4paper]{moderncv}
\moderncvtheme[blue]{classic}                
\usepackage[utf8]{inputenc}
\usepackage{eurosym}
\usepackage[scale=0.8]{geometry}
\usepackage{color}
\geometry{top=2.0cm, bottom=2.0cm, left=1.5cm, right=1.5cm}

\firstname{François}
\familyname{MONNIOT}
\title{\Large Élève Ingénieur Télécom \mbox{SudParis}}
\address{\bfseries6 rue de la République}{60600 Erquery}
\mobile{+33 (0)6.37.18.81.24}
\email{francoismonniot@gmail.com}
%homepage{www.francois-monniot.fr}
\extrainfo{23 ans, Détenteur du permis B}

\begin{document}
\maketitle

\section{Formation}
\cventry{2011 -- }{Élève ingénieur}{Télécom SudParis}{Institut Mines-Télécom}{Évry}{}
\cventry{2010 -- 2011}{Classes préparatoires aux grandes écoles ATS}{Lycée Marie Curie}{Nogent sur Oise}{}{}
  \cventry{2008 -- 2010}{BTS Controle Informatique et Régulation Automatique}{Lycée Marie Curie}{Nogent sur Oise}{}{}
\cventry{2008}{Baccalauréat STL mention Très Bien}{Lycée Marie Curie}{Nogent sur Oise}{}{spécialité PLPI}

\section{Expériences professionnelles}
\cventry{Juin 2012 à\\ Aout 2012}{Amélioration du système de mise à jour des postes SNCF et mise en place d'un outil de gestion du parc informatique.}{}{}{}{SNCF, Direction de l'ingenierie Nord-Paris.}
\cventry{Mai 2009 à\\ Aout 2009}{Mise en place d'une supervision et migration d'un système d'automate industriel}{}{}{}{Sucrerie de Chevrières, 60617}


\section{Expériences associatives}
\cventry{2011 -- }{Trésorier 2012-2013, Développeur, Administrateur réseau}{Association MiNET}{}{Association MiNET,FAI associatif (700 adhérents \& budget de 40k\officialeuro /an) du campus des écoles Télécom SudParis et Télécom Ecole de Management.}{Serveurs virtualisés avec OpenVZ/KVM sur un cluster Proxmox, matériel Cisco, interface de gestion en Ruby on Rails.}
\cventry{2011 -- 2012}{Développeur C++/Python}{Club INTech}{}{Club de robotique de l'école d'ingénieurs \mbox{Télécom SudParis}\small .}{9ème à la coupe de France de robotique de 2012.}

\cventry{2009 -- }{Membre actif}{Association ASA Tir à l'arc}{}{Association sportive de tir à l'arc dans l'oise}{}

\section{Projets}
\cventry{2012 -- }{Développement d'un honeypot SSH en C}{détection et analyse des attaques SSH}{}{}{}{}


\section{Compétences}
\cvline{Dév. Système}{C, C++, Python, Ruby, Java}
\cvline{Dév. Web}{PHP, Symfony2, HTML5/CSS3, Javascript, jQuery, Ruby on Rails}
\cvline{Dév. Mobile}{Android, Web}
\cvline{OS}{GNU/Linux, Microsoft Windows}
\cvline{Sécurité}{Web (XSS, Injections SQL, CSRF, ...), Honeypots}
\cvline{Divers}{SQL, Git, Subversion, \LaTeX}

\section{Langues}
\cvlanguage{Anglais}{Lu, parlé, écrit}{(niveau européen B1)}
\cvlanguage{Allemand}{Débutant}{}

\section{Centres d'intér\^{e}t}
\cvline{Loisirs}{Lecture, Musique}
\cvline{Sport}{Tir à l'arc}

\end{document}
